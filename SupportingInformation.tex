% Options for packages loaded elsewhere
\PassOptionsToPackage{unicode}{hyperref}
\PassOptionsToPackage{hyphens}{url}
%
\documentclass[
]{article}
\usepackage{amsmath,amssymb}
\usepackage{iftex}
\ifPDFTeX
  \usepackage[T1]{fontenc}
  \usepackage[utf8]{inputenc}
  \usepackage{textcomp} % provide euro and other symbols
\else % if luatex or xetex
  \usepackage{unicode-math} % this also loads fontspec
  \defaultfontfeatures{Scale=MatchLowercase}
  \defaultfontfeatures[\rmfamily]{Ligatures=TeX,Scale=1}
\fi
\usepackage{lmodern}
\ifPDFTeX\else
  % xetex/luatex font selection
\fi
% Use upquote if available, for straight quotes in verbatim environments
\IfFileExists{upquote.sty}{\usepackage{upquote}}{}
\IfFileExists{microtype.sty}{% use microtype if available
  \usepackage[]{microtype}
  \UseMicrotypeSet[protrusion]{basicmath} % disable protrusion for tt fonts
}{}
\makeatletter
\@ifundefined{KOMAClassName}{% if non-KOMA class
  \IfFileExists{parskip.sty}{%
    \usepackage{parskip}
  }{% else
    \setlength{\parindent}{0pt}
    \setlength{\parskip}{6pt plus 2pt minus 1pt}}
}{% if KOMA class
  \KOMAoptions{parskip=half}}
\makeatother
\usepackage{xcolor}
\usepackage[vmargin=1in,hmargin=1in]{geometry}
\usepackage{longtable,booktabs,array}
\usepackage{calc} % for calculating minipage widths
% Correct order of tables after \paragraph or \subparagraph
\usepackage{etoolbox}
\makeatletter
\patchcmd\longtable{\par}{\if@noskipsec\mbox{}\fi\par}{}{}
\makeatother
% Allow footnotes in longtable head/foot
\IfFileExists{footnotehyper.sty}{\usepackage{footnotehyper}}{\usepackage{footnote}}
\makesavenoteenv{longtable}
\usepackage{graphicx}
\makeatletter
\def\maxwidth{\ifdim\Gin@nat@width>\linewidth\linewidth\else\Gin@nat@width\fi}
\def\maxheight{\ifdim\Gin@nat@height>\textheight\textheight\else\Gin@nat@height\fi}
\makeatother
% Scale images if necessary, so that they will not overflow the page
% margins by default, and it is still possible to overwrite the defaults
% using explicit options in \includegraphics[width, height, ...]{}
\setkeys{Gin}{width=\maxwidth,height=\maxheight,keepaspectratio}
% Set default figure placement to htbp
\makeatletter
\def\fps@figure{htbp}
\makeatother
\setlength{\emergencystretch}{3em} % prevent overfull lines
\providecommand{\tightlist}{%
  \setlength{\itemsep}{0pt}\setlength{\parskip}{0pt}}
\setcounter{secnumdepth}{-\maxdimen} % remove section numbering
% definitions for citeproc citations
\NewDocumentCommand\citeproctext{}{}
\NewDocumentCommand\citeproc{mm}{%
  \begingroup\def\citeproctext{#2}\cite{#1}\endgroup}
\makeatletter
 % allow citations to break across lines
 \let\@cite@ofmt\@firstofone
 % avoid brackets around text for \cite:
 \def\@biblabel#1{}
 \def\@cite#1#2{{#1\if@tempswa , #2\fi}}
\makeatother
\newlength{\cslhangindent}
\setlength{\cslhangindent}{1.5em}
\newlength{\csllabelwidth}
\setlength{\csllabelwidth}{3em}
\newenvironment{CSLReferences}[2] % #1 hanging-indent, #2 entry-spacing
 {\begin{list}{}{%
  \setlength{\itemindent}{0pt}
  \setlength{\leftmargin}{0pt}
  \setlength{\parsep}{0pt}
  % turn on hanging indent if param 1 is 1
  \ifodd #1
   \setlength{\leftmargin}{\cslhangindent}
   \setlength{\itemindent}{-1\cslhangindent}
  \fi
  % set entry spacing
  \setlength{\itemsep}{#2\baselineskip}}}
 {\end{list}}
\usepackage{calc}
\newcommand{\CSLBlock}[1]{\hfill\break\parbox[t]{\linewidth}{\strut\ignorespaces#1\strut}}
\newcommand{\CSLLeftMargin}[1]{\parbox[t]{\csllabelwidth}{\strut#1\strut}}
\newcommand{\CSLRightInline}[1]{\parbox[t]{\linewidth - \csllabelwidth}{\strut#1\strut}}
\newcommand{\CSLIndent}[1]{\hspace{\cslhangindent}#1}
\usepackage{pdflscape,booktabs}
\newcommand{\blandscape}{\begin{landscape}}
\newcommand{\elandscape}{\end{landscape}}
\ifLuaTeX
  \usepackage{selnolig}  % disable illegal ligatures
\fi
\IfFileExists{bookmark.sty}{\usepackage{bookmark}}{\usepackage{hyperref}}
\IfFileExists{xurl.sty}{\usepackage{xurl}}{} % add URL line breaks if available
\urlstyle{same}
\hypersetup{
  pdftitle={Supplementary Material for `The response of trophic interaction networks to multiple stressors in a marine latitudinal gradient of the Southern Hemisphere'},
  hidelinks,
  pdfcreator={LaTeX via pandoc}}

\title{Supplementary Material for `The response of trophic interaction
networks to multiple stressors in a marine latitudinal gradient of the
Southern Hemisphere'}
\author{}
\date{\vspace{-2.5em}}

\begin{document}
\maketitle

Tomás I. Marina\textsuperscript{1}, Leonardo A.
Saravia\textsuperscript{1,2,*}, Iara D. Rodriguez\textsuperscript{3},
Manuela Funes\textsuperscript{4}, Georgina Cordone\textsuperscript{5},
Santiago R. Doyle\textsuperscript{3,6}, Anahí
Silvestro\textsuperscript{6}, David E. Galván\textsuperscript{5},
Susanne Kortsch\textsuperscript{7} \& Fernando Momo\textsuperscript{3,6}

\textsuperscript{1} Centro Austral de Investigaciones Científicas
(CADIC-CONICET), Ushuaia, Argentina;

\textsuperscript{2} Instituto de Ciencias Polares, Ambiente y Recursos
Naturales, Universidad Nacional de Tierra del Fuego (UNTdF), Ushuaia,
Argentina;

\textsuperscript{3} Instituto de Ciencias, Universidad Nacional de
General Sarmiento (UNGS), Los Polvorines, Argentina;

\textsuperscript{4} Instituto de Investigaciones Marinas y Costeras
(IIMyC-CONICET), Mar del Plata, Argentina;

\textsuperscript{5} Centro Para el Estudio de Sistemas Marinos
(CESIMAR-CONICET), Puerto Madryn, Argentina;

\textsuperscript{6} Instituto de Ecología y Desarrollo Sustentable
(INEDES-CONICET-UNLu), Luján, Argentina;

\textsuperscript{7} Tvärminne Zoological Station, University of
Helsinki, Hanko, Finland.

* corresponding author: Leonardo A. Saravia. Centro Austral de
Investigaciones Científicas (CADIC-CONICET), Ushuaia, Argentina.
\href{mailto:lasaravia@untdf.edu.ar}{\nolinkurl{lasaravia@untdf.edu.ar}}

\newpage

Table S1. Node-level properties for species being impacted by
environmental and/or anthropogenic stressors (see Table 2 in the article
for details) in six study areas along the southwest Atlantic to
Antarctic gradient (45 - 78ºS). Degree: total number of interactions;
relative abundance: high (+++), medium (+), low (-), unknown (?).

\begin{longtable}[]{@{}
  >{\raggedright\arraybackslash}p{(\columnwidth - 10\tabcolsep) * \real{0.1809}}
  >{\raggedright\arraybackslash}p{(\columnwidth - 10\tabcolsep) * \real{0.2340}}
  >{\raggedright\arraybackslash}p{(\columnwidth - 10\tabcolsep) * \real{0.1383}}
  >{\raggedright\arraybackslash}p{(\columnwidth - 10\tabcolsep) * \real{0.1277}}
  >{\raggedright\arraybackslash}p{(\columnwidth - 10\tabcolsep) * \real{0.1383}}
  >{\raggedright\arraybackslash}p{(\columnwidth - 10\tabcolsep) * \real{0.1596}}@{}}
\toprule\noalign{}
\begin{minipage}[b]{\linewidth}\raggedright
\textbf{Food web}
\end{minipage} & \begin{minipage}[b]{\linewidth}\raggedright
\textbf{Node / trophic species}
\end{minipage} & \begin{minipage}[b]{\linewidth}\raggedright
\textbf{Degree}
\end{minipage} & \begin{minipage}[b]{\linewidth}\raggedright
\textbf{Trophic level}
\end{minipage} & \begin{minipage}[b]{\linewidth}\raggedright
\textbf{Omnivory level}
\end{minipage} & \begin{minipage}[b]{\linewidth}\raggedright
\textbf{Relative abundance}
\end{minipage} \\
\midrule\noalign{}
\endhead
\bottomrule\noalign{}
\endlastfoot
\textbf{Beagle Channel} & & & & & \\
& \emph{Eleginops maclovinus} & 42 & 3.11 & 0.752 & ? \\
& \emph{Patagonotothen tessellata} & 35 & 3.29 & 0.721 & ++ \\
& \emph{Paralomis granulosa} & 26 & 3.28 & 0.277 & ? \\
& \emph{Patagonotothen cornucola} & 23 & 3.23 & 0.495 & +++ \\
& \emph{Nacella magellanica} & 22 & 2.46 & 0.397 & ? \\
& \emph{Mytilus edulis chilensis} & 21 & 2.42 & 0.368 & ? \\
& \emph{Sprattus fuegensis} & 17 & 3.05 & 0.422 & ++++ \\
& \emph{Harpagifer bispinis} & 15 & 3.51 & 0.145 & ? \\
& \emph{Paranotothenia magellanica} & 14 & 3.42 & 0.450 & +++ \\
& \emph{Odontesthes nigricans} & 11 & 3.50 & 0.428 & ? \\
& \emph{Patagonotothen sima} & 7 & 3.44 & 0.115 & ++ \\
& \emph{Grimothea gregaria} & 71 & 2.41 & 0.434 & ++ \\
\textbf{Burdwood Bank} & & & & & \\
& \emph{Sprattus fuegensis} & 42 & 3.45 & 0.688 & ++ \\
& \emph{Patagonotothen ramsayi} & 68 & 3.48 & 0.492 & +++ \\
& \emph{Cottoperca trigloides} & 20 & 4.16 & 0.742 & ++ \\
& \emph{Dissostichus eleginoides} & 49 & 4.01 & 0.797 & + \\
& \emph{Dorytheuthis gahi} & 38 & 3.45 & 0.896 & ? \\
& \emph{Henricia obesa} & 5 & 2.19 & 0.160 & ? \\
& \emph{Odontaster penicillatus} & 7 & 2.94 & 0.148 & ? \\
& \emph{Patagonotothen guntheri} & 63 & 3.39 & 0.390 & ++ \\
& \emph{Coelorinchus fasciatus} & 16 & 3.48 & 0.446 & ++ \\
& \emph{Coelorinchus marinii} & 8 & 3.32 & 0.658 & + \\
& \emph{Macrourus carinatus} & 17 & 4.13 & 0.455 & +++ \\
& \emph{Macrourus holotrachys} & 16 & 3.79 & 0.528 & +++ \\
& \emph{Tedania sp} & 2 & 2.00 & 0.000 & +++ \\
& \emph{Fusitriton magellanicus} & 8 & 3.11 & 0.017 & ++ \\
& \emph{Adelomelon ancilla} & 5 & 3.46 & 0.220 & + \\
& \emph{Zygochlamys patagonica} & 18 & 2.28 & 0.250 & ++ \\
& \emph{Hiatella sp} & 7 & 2.00 & 0.000 & ? \\
& \emph{Eurypodius latreillii} & 2 & 3.00 & 0.000 & + \\
& \emph{Sympagurus dimorphus} & 4 & 2.00 & 0.000 & ? \\
& \emph{Grimothea gregaria} & 32 & 2.50 & 0.250 & ? \\
& \emph{Libidoclaea granaria} & 7 & 2.00 & 0.000 & ? \\
& \emph{Alcyonium antarcticum} & 5 & 2.00 & 0.000 & ? \\
& \emph{Convexella magelhaenica} & 14 & 2.70 & 0.410 & ? \\
& \emph{Anthoptilum grandiflorum} & 9 & 2.78 & 0.395 & ? \\
& \emph{Primnoella scotiae} & 13 & 2.78 & 0.395 & ? \\
& \emph{Primnoisis antarctica} & 14 & 2.64 & 0.413 & ? \\
& \emph{Actinostola crassicornis} & 7 & 2.83 & 0.138 & + \\
& \emph{Brachiopoda spp} & 24 & 2.00 & 0.000 & ? \\
& \emph{Acodontaster sp} & 2 & 2.00 & 0.000 & ? \\
& \emph{Anasterias antarctica} & 2 & 3.57 & 0.000 & ++ \\
& \emph{Ceramaster patagonicus} & 4 & 3.53 & 0.096 & ? \\
& \emph{Henricia obesa} & 5 & 2.20 & 0.160 & ? \\
& \emph{Cosmasterias lurida} & 12 & 3.24 & 0.418 & ? \\
& \emph{Glabraster antarctica} & 4 & 2.00 & 0.000 & - \\
& \emph{Pteraster affinis} & 3 & 3.00 & 0.000 & ? \\
& \emph{Perknaster sp} & 2 & 3.08 & 0.000 & ? \\
& \emph{Ctenodiscus australis} & 2 & 2.00 & 0.000 & ? \\
& \emph{Odontaster sp} & 12 & 2.98 & 0.279 & ? \\
& \emph{Gorgonocephalus chilensis} & 4 & 3.59 & 0.009 & ? \\
& \emph{Ophiacantha vivipara} & 7 & 3.44 & 0.052 & ? \\
& \emph{Pseudechinus magellanicus} & 9 & 2.50 & 0.250 & ? \\
& \emph{Sterechinus agassizii} & 6 & 3.19 & 0.009 & ? \\
& \emph{Austrocidaris canaliculata} & 7 & 3.17 & 0.028 & - \\
& \emph{Chaetopterus antarcticus} & 1 & 2.00 & 0.000 & - \\
& \emph{Cnemidocarpa spp} & 12 & 2.10 & 0.090 & ? \\
& \emph{Daption capense} & 11 & 3.55 & 0.510 & ? \\
& \emph{Thalassarche melanophris} & 22 & 3.83 & 0.983 & ? \\
& \emph{Macronectes giganteus} & 7 & 3.93 & 0.903 & ? \\
& \emph{Thalassarche chrysostoma} & 16 & 4.06 & 0.906 & ? \\
& \emph{Diomedea epomophora} & 6 & 3.97 & 1.022 & ? \\
& \emph{Macronectes halli} & 6 & 4.00 & 1.017 & ? \\
& \emph{Procellaria aequinoctialis} & 8 & 4.06 & 0.782 & ? \\
\textbf{Scotia Sea, North \& South} & & & & & \\
& \emph{Euphausia superba} & 146 & 3.12 & 0.928 & +++ \\
& \emph{Dissostichus eleginoides} & 29 & 5.99 & 0.249 & +++ \\
& \emph{Krefftichthys anderssoni} & 157 & 4.50 & 1.010 & ++ \\
& \emph{Protomyctophum bolini} & 135 & 4.09 & 0.773 & + \\
& \emph{Electrona antarctica} & 192 & 4.50 & 1.000 & +++ \\
& \emph{Gymnoscopelus nicholsi} & 134 & 4.07 & 0.781 & - \\
& \emph{Gymnoscopelus braueri} & 185 & 4.53 & 0.978 & ++ \\
& \emph{Limacina retroversa} & 35 & 2.00 & 0.000 & +++ \\
\textbf{Potter Cove (Antarctica)} & & & & & \\
& Benthic Diatomea & 36 & 1.00 & 0.000 & - \\
& Copepoda & 27 & 2.60 & 0.720 & - \\
& Phytoplankton & 23 & 1.00 & 0.000 & - \\
& Zooplankton & 17 & 2.80 & 0.640 & - \\
& Porifera & 18 & 2.00 & 8.043 & - \\
& \emph{Euphausia superba} & 17 & 3.42 & 0.734 & + \\
& Salpidae & 13 & 3.28 & 1.198 & - \\
& Ascidiacea & 9 & 2.45 & 0.607 & +++ \\
& \emph{Malacobelemnon daytoni} & 2 & 2.90 & 0.810 & + \\
Macroalgae & \emph{Palmaria decipiens} & 11 & 1.00 & 0.000 & + \\
& \emph{Gigartina skottsbergii} & 8 & 1.00 & 0.000 & + \\
& \emph{Desmarestia menziesii} & 7 & 1.00 & 0.000 & +++ \\
& \emph{Desmarestia anceps} & 6 & 1.00 & 0.000 & +++ \\
& \emph{Desmarestia antarctica} & 6 & 1.00 & 0.000 & + \\
& \emph{Iridaea cordata} & 6 & 1.00 & 0.000 & + \\
& \emph{Plocamium cartilagineum} & 6 & 1.00 & 0.000 & - \\
& \emph{Myriogramme manginii} & 4 & 1.00 & 0.000 & - \\
& \emph{Adenocystis utricularis} & 3 & 1.00 & 0.000 & + \\
& \emph{Georgiella confluens} & 3 & 1.00 & 0.000 & - \\
& \emph{Ascoseira mirabilis} & 2 & 1.00 & 0.000 & + \\
& \emph{Phaeurus antarcticus} & 2 & 1.00 & 0.000 & - \\
& \emph{Curdiea racovitzae} & 2 & 1.00 & 0.000 & - \\
& \emph{Callophyllis atrosanguinea} & 1 & 1.00 & 0.000 & - \\
& \emph{Neuroglossum delesseriae} & 1 & 1.00 & 0.000 & - \\
& \emph{Pantoneura plocamioides} & 1 & 1.00 & 0.000 & - \\
& \emph{Picconiella plumosa} & 1 & 1.00 & 0.000 & - \\
\textbf{Weddell Sea (Antarctica)} & & & & & \\
& \emph{Pleuragramma antarcticum} & 69 & 3.58 & 0.440 & +++ \\
& \emph{Chionodraco myersi} & 37 & 4.09 & 0.356 & ++ \\
& \emph{Pagetopsis maculatus} & 37 & 4.09 & 0.356 & + \\
& \emph{Gymnoscopelus braueri} & 62 & 3.53 & 0.274 & ? \\
& \emph{Dissostichus mawsoni} & 87 & 4.12 & 0.560 & ? \\
& Bacillariophyceae & 81 & 1.00 & 0.000 & +++ \\
& Silicoflagellates & 30 & 1.00 & 0.000 & ++ \\
& Foraminifera & 93 & 2.00 & 0.000 & ++ \\
& \emph{Euphausia superba} & 163 & 2.27 & 0.385 & +++ \\
& \emph{Salpa thompsoni} & 108 & 2.28 & 0.347 & ++ \\
& \emph{Leptonychotes weddellii} & 59 & 4.86 & 0.204 & ? \\
& \emph{Lobodon carcinophagus} & 28 & 4.24 & 0.553 & +++ \\
& \emph{Hydrurga leptonyx} & 67 & 4.72 & 0.346 & ? \\
& \emph{Arctocephalus gazella} & 61 & 4.67 & 0.331 & ? \\
& \emph{Thalassoica antarctica} & 19 & 4.32 & 0.429 & +++ \\
& \emph{Pagodroma nivea} & 11 & 4.21 & 0.209 & ++ \\
& \emph{Sterna paradisaea} & 7 & 4.04 & 0.347 & ++ \\
& \emph{Aptenodytes forsteri} & 53 & 4.78 & 0.211 & ++ \\
& \emph{Pygoscelis adeliae} & 7 & 3.78 & 0.282 & +++ \\
& \emph{Megaptera novaeangliae} & 4 & 3.26 & 0.014 & ? \\
& Hexactinellida & 49 & 2.00 & 0.000 & +++ \\
\textbf{San Jorge Gulf} & & & & & \\
& The species of the San Jorge Gulf are not listed because the number of
stressed species reported is too high. At a functional group level of
resolution, the main impacted groups are: macroinvertebrates, fish, and
seabirds. & & & & \\
\end{longtable}

\newpage
\scriptsize

Table S2. Node and network-level properties used to build hypotheses on
the stressors' effects.

\begin{longtable}[]{@{}
  >{\raggedright\arraybackslash}p{(\columnwidth - 6\tabcolsep) * \real{0.1963}}
  >{\raggedright\arraybackslash}p{(\columnwidth - 6\tabcolsep) * \real{0.2617}}
  >{\raggedright\arraybackslash}p{(\columnwidth - 6\tabcolsep) * \real{0.3925}}
  >{\raggedright\arraybackslash}p{(\columnwidth - 6\tabcolsep) * \real{0.1495}}@{}}
\toprule\noalign{}
\begin{minipage}[b]{\linewidth}\raggedright
\textbf{Property}
\end{minipage} & \begin{minipage}[b]{\linewidth}\raggedright
\textbf{Definition}
\end{minipage} & \begin{minipage}[b]{\linewidth}\raggedright
\textbf{Relevance for stressor effects}
\end{minipage} & \begin{minipage}[b]{\linewidth}\raggedright
\textbf{Reference}
\end{minipage} \\
\midrule\noalign{}
\endhead
\bottomrule\noalign{}
\endlastfoot
\emph{Node-level} & & & \\
Degree & Number of feeding interactions in which the species
participates as prey and/or predator. & Perturbations to high-degree
species may have more significant effects on the food web robustness to
perturbations than low-degree species. & Dunne et al. (2002b); Jordán et
al. (2007) \\
Trophic position & Place in the food web relative to the basal resources
that support the community. Classifies species in: basal, intermediate
and top. & Perturbations on basal resources, intermediate species and
top predators are expected to have large effects on the rest of their
communities if ecosystem control is bottom-up, wasp-waist or top-down,
respectively. & Williams and Martinez (2000); Thompson et al. (2007) \\
Omnivory & Consumer resource use across trophic levels. & High-omnivore
species (generalists) are more flexible than low-omnivore species to
diet changes. & Thompson et al. (2007) \\
Relative abundance & Species' density in proportion to the other species
of the food web. & Perturbations on abundant (dominant) species are
expected to have large effects on the stability and energy flux of the
food and ecosystem, respectively. & Nilsson and McCann (2016) \\
\emph{Network-level} & & & \\
Conectance & Proportion of actual interactions among possible ones. &
Estimator of community sensitivity to stressors. High connectance gives
resistance and resilience to the food web. & Dunne et al. (2002a) \\
Path length & Average distance, accounted by the number of interactions,
between any pair of species. & Short distances enhance rapid and broad
propagation of perturbations. & Albert and Barabási (2002) \\
Mean TL & Average of all species' trophic position contained in the food
web. & Influences the magnitude and efficiency of trophic transfer. A
higher mean food chain length reflects increased energy availability and
productivity. & Duffy et al. (2007); Olivier et al. (2019) \\
Omnivory & Proportion of species that feed at different trophic levels.
& It provides trophic flexibility to an ecosystem. Reduces probability
of trophic cascades. & Kratina et al. (2012) \\
\end{longtable}

\normalsize

\section*{References}\label{references}
\addcontentsline{toc}{section}{References}

\phantomsection\label{refs}
\begin{CSLReferences}{1}{0}
\bibitem[\citeproctext]{ref-Albert2002}
Albert, R., and Barabási, A.-L. 2002. Statistical mechanics of complex
networks. Reviews of Modern Physics \textbf{74}(1): 47--97. American
Physical Society.
doi:\href{https://doi.org/10.1103/RevModPhys.74.47}{10.1103/RevModPhys.74.47}.

\bibitem[\citeproctext]{ref-Duffy2007}
Duffy, J.E., Cardinale, B.J., France, K.E., McIntyre, P.B., Thébault,
E., and Loreau, M. 2007. The functional role of biodiversity in
ecosystems: Incorporating trophic complexity. Ecology Letters
\textbf{10}(6): 522--538.
doi:\href{https://doi.org/10.1111/j.1461-0248.2007.01037.x}{10.1111/j.1461-0248.2007.01037.x}.

\bibitem[\citeproctext]{ref-Dunne2002}
Dunne, J.A., Williams, R.J., and Martinez, N.D. 2002a. Network structure
and biodiversity loss in food webs: Robustness increases with
connectance. Ecology Letters \textbf{5}(4): 558--567.
doi:\href{https://doi.org/10.1046/j.1461-0248.2002.00354.x}{10.1046/j.1461-0248.2002.00354.x}.

\bibitem[\citeproctext]{ref-Dunne2002a}
Dunne, J.A., Williams, R.J., and Martinez, N.D. 2002b. Food-web
structure and network theory: {The} role of connectance and size.
Proceedings of the National Academy of Sciences \textbf{99}(20):
12917--12922. Proceedings of the National Academy of Sciences.
doi:\href{https://doi.org/10.1073/pnas.192407699}{10.1073/pnas.192407699}.

\bibitem[\citeproctext]{ref-Jordan2007}
Jordán, F., Benedek, Z., and Podani, J. 2007. Quantifying positional
importance in food webs: {A} comparison of centrality indices.
Ecological Modelling \textbf{205}(1): 270--275.
doi:\href{https://doi.org/10.1016/j.ecolmodel.2007.02.032}{10.1016/j.ecolmodel.2007.02.032}.

\bibitem[\citeproctext]{ref-Kratina2012}
Kratina, P., LeCraw, R.M., Ingram, T., and Anholt, B.R. 2012. Stability
and persistence of food webs with omnivory: {Is} there a general
pattern? Ecosphere \textbf{3}(6): art50.
doi:\href{https://doi.org/10.1890/ES12-00121.1}{10.1890/ES12-00121.1}.

\bibitem[\citeproctext]{ref-Nilsson2016}
Nilsson, K.A., and McCann, K.S. 2016. Interaction strength
revisited---clarifying the role of energy flux for food web stability.
Theoretical Ecology \textbf{9}(1): 59--71.
doi:\href{https://doi.org/10.1007/s12080-015-0282-8}{10.1007/s12080-015-0282-8}.

\bibitem[\citeproctext]{ref-Olivier2019}
Olivier, P., Frelat, R., Bonsdorff, E., Kortsch, S., Kröncke, I.,
Möllmann, C., Neumann, H., Sell, A.F., and Nordström, M.C. 2019.
Exploring the temporal variability of a food web using long-term
biomonitoring data. Ecography \textbf{42}(12): 2107--2121.
doi:\href{https://doi.org/10.1111/ecog.04461}{10.1111/ecog.04461}.

\bibitem[\citeproctext]{ref-Thompson2007}
Thompson, R.M., Hemberg, M., Starzomski, B.M., and Shurin, J.B. 2007.
Trophic {Levels} and {Trophic Tangles}: {The Prevalence} of {Omnivory}
in {Real Food Webs}. Ecology \textbf{88}(3): 612--617.
doi:\href{https://doi.org/10.1890/05-1454}{10.1890/05-1454}.

\bibitem[\citeproctext]{ref-Williams2000}
Williams, R.J., and Martinez, N.D. 2000. Simple rules yield complex food
webs. Nature \textbf{404}(6774): 180--183. Nature Publishing Group.
doi:\href{https://doi.org/10.1038/35004572}{10.1038/35004572}.

\end{CSLReferences}

\end{document}
